\section{Conclusions}
\label{sec:conclusions}
In this paper, we present the Hybrid-Float6 quantization for floating-point CNN hardware acceleration. Feature maps and weights are represented by 32-bit and 6-bit floating-point, respectively. The 6-bit floating-point format is composed of 1-bit sign, 4-bit exponent, and 1-bit mantissa. The 1-bit mantissa enables low-power multiply-accumulate implementations by reducing the mantissa multiplication to a multiplexer-adder operation. We exploit the intrinsic error tolerance of neural networks to further reduce the hardware design with approximation. This approach improves latency, hardware area, and energy consumption. To preserve accuracy, we introduce a quantization aware training method that, in some cases, improves accuracy. We present a lightweight tensor processor implementing a pipelined vector dot-product. For ML compatibility/portability, the 6-bit FP is wrapped in the standard floating-point format, which is automatically extracted by the proposed hardware. The hardware/software architecture is compatible with TensorFlow Lite. We evaluate the applicability of our approach with a CNN-regression model for anomaly localization in a structural health monitoring application based on acoustic emissions. The embedded hardware/software framework is demonstrated on XC7Z007S as the smallest Zynq-7000 SoC. The proposed hardware achieves a peak power efficiency and acceleration on convolution layers of $5.7$ GFLOPS/s/W and $48.3\times$, respectively.