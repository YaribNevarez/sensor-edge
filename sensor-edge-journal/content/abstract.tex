\title {CNN Sensor Analytics with Hybrid-Float6 on Low-Power Resource-Constrained Embedded FPGAs.}


\tfootnote{This work is funded by the Consejo Nacional de Ciencia y Tecnologia - CONACYT}

\markboth
{Author \headeretal: Preparation of Papers for IEEE TRANSACTIONS and JOURNALS}
{Author \headeretal: Preparation of Papers for IEEE TRANSACTIONS and JOURNALS}

\corresp{Corresponding author: Yarib Nevarez (e-mail: nevarez@item.uni-bremen.de).}

\begin{abstract}
The use of artificial intelligence (AI) in sensor analytics is entering a new era based on the use of ubiquitous embedded connected devices. This transformation requires the adoption of design techniques that reconcile accurate results with sustainable system architectures. As such, improving efficiency of AI hardware engines must be considered. In this paper, we present the Hybrid-Float6 (HF6) quantization on shallow CNNs for sensor data analytics and its dedicated hardware design. This approach reduces over-fit on feature extraction and improves generalization. As dedicated hardware design, we propose a fully customizable tensor processor (TP) implementing a pipelined vector dot-product with HF6. This approach reduces energy consumption and resource utilization. The proposed embedded hardware/software architecture is unified with TensorFlow Lite. We evaluate the applicability of our approach with a data analytics application for structural health monitoring (SHM) for anomaly localization. The embedded hardware/software framework is demonstrated on XC7Z007S as the smallest and most inexpensive Zynq SoC device.
\end{abstract}

\begin{keywords}
Convolutional neural networks, structural health monitoring, hardware accelerator, TensorFlow Lite, embedded systems, FPGA, custom floating-point
\end{keywords}

\titlepgskip=-15pt

\maketitle