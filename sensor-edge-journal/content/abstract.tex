\title {SensorEdge: Design Exploration Framework for Near-Sensor Analytics on Resource-Constrained Embedded FPGAs}


\tfootnote{This work is funded by the Consejo Nacional de Ciencia y Tecnologia - CONACYT}

\markboth
{Author \headeretal: Preparation of Papers for IEEE TRANSACTIONS and JOURNALS}
{Author \headeretal: Preparation of Papers for IEEE TRANSACTIONS and JOURNALS}

\corresp{Corresponding author: Yarib Nevarez (e-mail: nevarez@item.uni-bremen.de).}

\begin{abstract}
The use of artificial intelligence (AI) in near-sensor analytic applications is entering a new era based on the use of ubiquitous embedded connected devices. This transformation requires the adoption of design techniques that reconcile accurate results with sustainable system architectures. As such, maximizing computational efficiency given limited hardware and power resources is increasingly being considered, and as a result, reduced-precision inference has emerged as a viable alternative to the IEEE 754 full precision floating-point arithmetic. In this paper, we present a design methodology for training, quantization, and deployment of convolutional neural networks (CNNs) with dedicated hardware acceleration targeting low-power and resource-limited embedded FPGAs. The key contributions of this work are the demonstration of custom reduced floating-point quantization and its dedicated hardware design for low-power resource-constrained data analytics applications. We propose a quantization aware training method for fine-tuning CNN-based models. This approach improves generalization and increases overall accuracy using less than 8-bits custom floating-point quantization on trainable parameters. As dedicated hardware design, we propose a fully customizable tensor processor (TP) implementing a pipelined vector dot-product with hybrid custom floating-point and logarithmic approximation. This approach reduces energy consumption and resource utilization preserving inference accuracy. The proposed embedded hardware/software architecture is unified with TensorFlow Lite. We demonstrate our framework implementing a CNN-based sensor analytic application for structural health monitoring (SHM) for anomaly localization. The embedded hardware/software framework is demonstrated on XC7Z007S as the smallest and most inexpensive Zynq SoC device. The TP achieves a peak runtime acceleration of 55X on Conv2D tensor operators, and power efficiency of 4.5 GFLOP/s/W with 55\% of hardware resource utilization.
\end{abstract}

\begin{keywords}
Convolutional neural networks, depthwise separable convolution, hardware accelerator, TensorFlow Lite, embedded systems, FPGA, custom floating-point, logarithmic computation, approximate computing
\end{keywords}

\titlepgskip=-15pt

\maketitle