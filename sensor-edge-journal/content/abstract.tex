\title {CNN Sensor Analytics with Hybrid-Float6 on Low-Power Resource-Constrained Embedded FPGAs.}


\tfootnote{This work is funded by the Consejo Nacional de Ciencia y Tecnologia - CONACYT}

\markboth
{Author \headeretal: Preparation of Papers for IEEE TRANSACTIONS and JOURNALS}
{Author \headeretal: Preparation of Papers for IEEE TRANSACTIONS and JOURNALS}

\corresp{Corresponding author: Yarib Nevarez (e-mail: nevarez@item.uni-bremen.de).}

\begin{abstract}
The use of artificial intelligence (AI) in sensor analytics applications is entering a new era based on the use of ubiquitous embedded connected devices. This transformation requires the adoption of design techniques that reconcile accurate results with sustainable system architectures. As such, maximizing computational efficiency given limited hardware and power resources is increasingly being considered, and as a result, reduced-precision inference has emerged as a viable alternative to the IEEE 754 full precision floating-point arithmetics. In this paper, we present the Hybrid-Float6 (HF6) quantization on shallow CNNs for sensor data analytics and its dedicated hardware design for low-power resource-constrained embedded FPGAs. This approach improves generalization by reducing over-fit on feature extraction. As dedicated hardware design, we propose a fully customizable tensor processor (TP) implementing a pipelined vector dot-product with HF6. This approach reduces energy consumption and resource utilization preserving inference accuracy. The proposed embedded hardware/software architecture is unified with TensorFlow Lite. We evaluate the applicability of our framework with a CNN-model and hardware design exploration for sensor analytics of anomaly localization based on regression in structural health monitoring (SHM). The embedded hardware/software framework is demonstrated on XC7Z007S as the smallest and most inexpensive Zynq SoC device. The TP achieves a peak runtime acceleration of 55X on Conv2D tensor operators, and power efficiency of 4.5 GFLOP/s/W with 55\% of hardware resource utilization.
\end{abstract}

\begin{keywords}
Convolutional neural networks, structural health monitoring, hardware accelerator, TensorFlow Lite, embedded systems, FPGA, custom floating-point
\end{keywords}

\titlepgskip=-15pt

\maketitle