\section{Background}
\label{sec:background}
\subsection{Floating-point Number Representation}
Representation of every numeric value, in any number system, is composed of an integer and a fractional part. The boundary that delimits them is called the radix point. The fixed-point format for representing numeric values derives its name from the fact that in this format, the radix point is fixed in a certain position. For integers this position is immediately to the right of the least significant digit.

In scientific computation, it is frequently necessary to represent very large and very small values. This is difficult to achieve using the fixed-point format because the bitwidth required to maintain both the desired precision and the desired range grows large. In such situations, floating point formats are used to represent real numbers. Floating point formats resemble scientific notation, such as $-3.2004\times10^{17}$. Every floating point number can be divided into three fields: sign $s$, exponent $e$, and fraction $f$. Using the binary number system, it is possible to represent any floating-point number as:

\begin{eqnarray} \label{eq:float}
(-1)^{2} \times 1.f \times 2^{e-BIAS}
\end{eqnarray}

Note that the exponent is biased, meaning that the stored value is shifted from 0 by a given value that depends on the bitwidth of the exponent field in the particular format. Also, the fraction represents the portion of the mantissa
to the right of the radix point, while the term mantissa refers to the fractional and integer part.

A natural tradeoff exists between smaller bitwidths requiring fewer hardware resources and higher bitwidths providing better precision. Also, within a given total bitwidth, it is possible to assign various combinations of bitwidths to the exponent and fraction fields, where wider exponents result in higher range and wider fractions result in better precision.

The most widely used format for floating point arithmetic is the IEEE standard 754 [1]. This standard details four floating point formats - basic and extended, each in single and double precision bitwidths. The IEEE single precision
format is the same as shown in Equation 1 with BIAS = 127, 8 bits for the exponent and 23 bits for the fraction, or a total of 32 bits. In IEEE format, numbers are normalized, and only the fractional part is stored.

Optimal implementations of algorithms frequently do not require the bitwidths specified by the IEEE standard. Often, much smaller bitwidths than those specified in the 754 standard are sufficient to provide the desired precision. Reduced bitwidth implementations require fewer resources and thus allow for more parallel implementations than using the full IEEE standard format. In custom hardware designs, it is possible, and indeed desirable, to have full control and flexibility over the exact floating point format implemented. Our library provides this flexibility.

\subsection{Conv2D tensor operation}
The \emph{Conv2D} tensor operation is described in \Equ{eq:conv2D}, where $h$ is the input feature map, $W$ is the convolution kernel (known as filter), and $b$ is the bias for the output feature map\cite{goodfellow2016deep}. We denote \emph{Conv} as \emph{Conv2D} operator.
\begin{eqnarray} \label{eq:conv2D}
Conv\left(W,h\right)_{i,j,o}=\sum_{k,l,m}^{K,L,M} h_{(i+k,j+l,m)} W_{(o,k,l,m)}+b_{o}
\end{eqnarray}